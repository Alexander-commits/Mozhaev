\documentclass[11pt]{article}

\usepackage[utf8]{inputenc}
\usepackage[russian]{babel}
\usepackage{algorithm}
\usepackage[noend]{algpseudocode}
\usepackage{biblatex}
\addbibresource{sample.bib}
\begin{document}
\linespread{1.3}

\newcommand{\eps}{\varepsilon}
\newcommand{\inprod}[1]{\left\langle #1 \right\rangle}

\newcommand{\handout}[5]{
	\noindent
	\begin{center}
		\framebox{
			\vbox{
				\hbox to 5.78in { {\bf Научно-исследовательская практика} \hfill #2 }
				\vspace{4mm}
				\hbox to 5.78in { {\Large \hfill #5  \hfill} }
				\vspace{2mm}
				\hbox to 5.78in { {\em #3 \hfill #4} }
			}
		}
	\end{center}
	\vspace*{4mm}
}

\newcommand{\lecture}[4]{\handout{#1}{#2}{#3}{Автор: #4}{Извлечение квадратных корней в конечном поле #1}}

\topmargin 0pt
\advance \topmargin by -\headheight
\advance \topmargin by -\headsep
\textheight 8.9in
\oddsidemargin 0pt
\evensidemargin \oddsidemargin
\marginparwidth 0.5in
\textwidth 6.5in

\parindent 0in
\parskip 1.5ex
\lecture{}{Лето 2020}{}{Можаев Александр}	

\section{Алгоритм Тонелли-Шенкса }

\begin{algorithm}
	
	\caption{Тонелли--Шенкса}
	\label{alg:AlgName}
	\textbf{Input:}\\
	$p$ -- нечётное простое число; \\
	$n$ -- целое число, являющееся квадратичным вычетом по модулю $p$. \\
	\textbf{Output:}\\
	Вычет $R$, удовлетворяющий сравнению $R^{2}\equiv n \pmod p$.
	
	\begin{algorithmic}[1]
		
		\State Выделить степени двойки из $p-1$, то есть положить $p-1= 2^{S}Q$, где Q-нечётно, {$S \geq 1$}. Если $S=1$, т.е. {$p\equiv 3 \pmod 4 $}, решение имеет вид $R\equiv \pm\: n^{p+1 \over {4}}$. 
		\State Положить {$S \geq 2$}. Выбрать произвольный квадратичный невычет $z$, т.е. $({z \over {p}}) = -1$. Задать $c \equiv z^{Q} \pmod p$.
		\State Положить $R\equiv n^{Q+1 \over {2}} \pmod p$, $t\equiv n^{Q} \pmod p$, $M = S$. 
		\State \textbf{Пока} $t \neq 1$:
			\begin{algorithmic}[1]
				\State \textbf{Если} $t\equiv 1 \pmod p$, то выход и ответ $R$.
				\State \textbf{Иначе для} $i = 1$ до $m$ искать $i$ такое, что $t^{2^{i}}\equiv 1\:(mod\:p)$ 
				\State Положить $b\equiv c^{2^{M-i-1}} \pmod p$, $R\equiv Rb \pmod p$, $t \equiv tb^{2} \pmod p$, $c \equiv b^{2} \pmod p$, $M = i$, вернуться к началу цикла.
			\end{algorithmic}	
	   \Statex Ответ: $R, p-R$.	
	\end{algorithmic}
\end{algorithm}

\newpage

\section{Анализ}
\subsection{Описание машины}

Лабораторная работа была сделана на {\textit Sage} -- системе компьютерной алгебры, покрывающей много областей математики, включая алгебру, комбинаторику, вычислительную математику и матанализ.

Характеристики компьютера, на котором проводилось тестирование:
\begin{itemize}
	\item 64-разрядная операционная система Windows 10 Pro.
	\item Процессор: Intel(R) Core(TM) i5-8400, ядро Coffee Lake S, 
	количество ядер и потоков: 6, базовая частота 2.80 ГГц,частота в режиме boost 4.0 ГГц.
	\item Оперативная память: 2x8 Gb DDR4 CORSAIR Vengeance,3000 МГц.
	\item Версия системы SageMath: 9.0.
\end{itemize}

\subsection{Сравнение}
Было произведено 3 теста с различными значениями, в ходе которых я пришел к выводу, что встроенный алгоритм извлечения квадратного корня по простому модулю работает для больших чисел медленнее,чем Алгоритм Тонелли-Шенкса. Сравнение времени работы двух алгоритмов представлено в виде таблицы: 
\\

\begin{center}
\begin{tabular}{|c|c|c|c|c|}
	\hline
	
	\multicolumn{2}{|c|}{Входные параметры} & \multicolumn{2}{|c|}{Время}\\
	\hline
	$Число n$ & $Модуль p$ & Тонелли-Шенкса & Встроенный\\[5pt] 
	\hline
	$46^{12}$ & \verb|next_prime|$(10^{86})$ & 594 мкс. & 92,5 мс.\\
	\hline
	$4166^{112}$ & \verb|next_prime|$(10^{459})$ & 3.73 мс. & 14.9 сек.\\
	\hline
    $486^{175}$ & \verb|next_prime|$(10^{1886})$ & 18.4 мс. & 1 мин. 26 сек.\\
	\hline	
\end{tabular}	
\end{center}
\begin{thebibliography}{1}	
	\bibitem{Cohen H.}
	Cohen H. A course in computational algebraic number theory.-- 1993, стр. 32--34.
	\bibitem{Wiki}
	Wikipedia: Tonelli–Shanks algorithm, https://en.wikipedia.org/wiki/Tonelli-Shanks\_algorithm, 15 April 2020.
\end{thebibliography}
\end{document}